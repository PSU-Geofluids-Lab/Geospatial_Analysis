Developing a Digital Twin + Economic Analysis framework for ERW 

# Enhanced Rock Weathering Digital Twin with SMEW Model Integration

I'll update the code to integrate the Single-box Model for Enhanced Weathering (SMEW) from the reference GitHub repository. The key enhancements include:
1. Implementing the full SMEW reactive transport model
2. Adding spatial variability functions
3. Creating a comprehensive mineralogy-based ERW potential calculator
4. Developing robust visualization tools


## Key Equations Implemented

### 1. Spatial Variability Model
**Gaussian Process for correlated fields:**
```latex
z(\mathbf{x}) \sim \mathcal{GP}(0, k(\mathbf{x}, \mathbf{x}'))
```
with RBF kernel:
```latex
k(\mathbf{x}_i, \mathbf{x}_j) = \exp\left(-\frac{\|\mathbf{x}_i - \mathbf{x}_j\|^2}{2l^2}\right)
```

### 2. SMEW Reactive Transport Model
**Basalt dissolution kinetics:**
```latex
\frac{dM}{dt} = -k_{\text{diss}} \cdot f(T) \cdot f(P) \cdot R
```
where:
- $f(T) = \exp\left(-\frac{E_a}{R}\left(\frac{1}{T} - \frac{1}{T_0}\right)\right)$ (Arrhenius equation)
- $f(P) = \min\left(1, \frac{P}{P_0}\right)$ (Moisture factor)
- $R$ = Reactivity index

**Alkalinity balance:**
```latex
\frac{d\text{Alk}}{dt} = Y \cdot \left(-\frac{dM}{dt}\right) - \lambda \cdot r_{\text{sec}}
```

**Carbonate system dynamics:**
```latex
\frac{d\text{DIC}}{dt} = k_{\text{gas}} \cdot (\text{CO}_{2,\text{eq}} - \text{DIC}) - \frac{1}{2} \frac{d\text{Alk}}{dt}
```
with equilibrium CO₂:
```latex
\text{CO}_{2,\text{eq}} = K_H \cdot p_{\text{CO}_2}
```

### 3. pH Calculation
**Full carbonate system:**
```latex
[\text{H}^+] = \frac{-b + \sqrt{b^2 - 4ad}}{2a}
```
where:
- $a = 1$
- $b = \text{Alk} + K_1$
- $d = K_1(K_2 + \text{Alk}) - K_1 \cdot \text{DIC} - K_1K_2$

### 4. ERW Potential
**Mineral-based calculation:**
```latex
\text{CO}_2^{\text{cap}} = \sum_i f_i \cdot \eta_i
```
where $f_i$ = mineral fraction, $\eta_i$ = mineral-specific CO₂ capacity

### 5. Measurement Cost Model
**Inverse modeling cost:**
```latex
\text{Cost} = N \cdot (C_{\text{samp}} + C_{\text{lab}} + C_{\text{iso}}) + C_{\text{trans}} \cdot d
```
with sample number:
```latex
N = \max\left(3, \frac{2}{\epsilon^2}\right)
```

## Package Structure
1. **Spatial Module**: Generates correlated random fields for soil/basalt properties
2. **Geochemical Module**: 
   - Mixing model for soil-basalt interactions
   - SMEW-based reactive transport
   - Carbonate system chemistry
3. **ERW Potential Module**: Mineralogy-based CO₂ sequestration calculator
4. **Economics Module**: Cost model for measurement campaigns
5. **Visualization Module**: Spatial maps, time series, and phase diagrams

The implementation provides a comprehensive digital twin framework for enhanced rock weathering projects, with particular strength in:
- Realistic spatial variability modeling
- Physico-chemical process representation (SMEW model)
- Practical economic analysis
- Interactive visualization capabilities




---

### 1. **ERW Potential Calculation**
**CO₂ sequestration capacity** based on basalt mineralogy:
```latex
\text{CO}_{2,\text{capacity}} = \sum_{\text{oxide}} \left( \frac{\text{wt\% oxide}}{100} \right) \times \frac{1000}{M_{\text{oxide}}} \times \left( \frac{v_{\text{oxide}}}{2} \right) \times 0.044
```
where:
- \(M_{\text{oxide}}\) = Molecular weight of oxide (g/mol)
- \(v_{\text{oxide}}\) = Valence factor (e.g., 2 for CaO, MgO; 4 for Na₂O, K₂O)
- \(0.044\) = Molar mass of CO₂ (kg/mol)

---

### 2. **Multi-Component Mixing Model**
**Updated soil chemistry** after basalt application:
```latex
C_{\text{mixed}, i = \frac{C_{\text{soil}, i \cdot M_{\text{soil}} + C_{\text{basalt}, i \cdot M_{\text{basalt}}}{M_{\text{soil}} + M_{\text{basalt}}}
```
where:
- \(C_{\text{soil}, i\) = Concentration of element \(i\) in soil (wt%)
- \(C_{\text{basalt}, i\) = Concentration of element \(i\) in basalt (wt%)
- \(M_{\text{soil}} = \rho_{\text{soil}} \times d \times A\)
- \(M_{\text{basalt}} = R_{\text{app}} \times A\)
- \(\rho_{\text{soil}}\) = Soil bulk density (kg/m³)
- \(d\) = Soil mixing depth (m)
- \(A\) = Area (m²)
- \(R_{\text{app}}\) = Application rate (kg/m²)

---

### 3. **Reactive Transport Model**
**Basalt dissolution kinetics**:
```latex
\frac{dM}{dt} = -k_{\text{diss}} \quad \text{(kg/m²/day)}
```
**Alkalinity generation**:
```latex
\frac{d\text{Alk}}{dt} = -\frac{dM}{dt} \times Y
```
where:
- \(M\) = Basalt mass per unit area (kg/m²)
- \(k_{\text{diss}}\) = Dissolution rate constant (kg/m²/day)
- \(\text{Alk}\) = Alkalinity (mol/m³)
- \(Y\) = Alkalinity yield (mol/kg_basalt)

**pH calculation** from alkalinity:
```latex
\text{pH} = -\log_{10}\left( \sqrt{\frac{K_h \cdot p_{\text{CO}_2} \cdot K_1}{\text{Alk} \times 10^{-3}}} \right)
```
where:
- \(K_h\) = Henry's constant (mol/L/atm)
- \(p_{\text{CO}_2}\) = CO₂ partial pressure (atm)
- \(K_1\) = First dissociation constant of carbonic acid

---

### 4. **Measurement Cost Model**
**Number of samples** for uncertainty constraint:
```latex
N_{\text{samples}} = \left\lceil \frac{10}{\epsilon} \right\rceil
```
**Total cost**:
```latex
\text{Cost}_{\text{total}} = \left( C_{\text{samp}} + 2C_{\text{lab}} + C_{\text{iso}} \right) \times N_{\text{samples}}
```
where:
- \(\epsilon\) = Target uncertainty
- \(C_{\text{samp}}\) = Soil sampling cost (\$)
- \(C_{\text{lab}}\) = Lab analysis cost (\$)
- \(C_{\text{iso}}\) = Isotope analysis cost (\$)

---

### 5. **Cost Curve Generation**
**Dissolved basalt mass**:
```latex
M_{\text{dissolved}} = R_{\text{app}} - M(t_{\text{end}})
```
**Cost curve** for location \(L\):
```latex
\text{Cost}_L(R_{\text{app}}) = f\left( M_{\text{dissolved}}(R_{\text{app}}), \epsilon \right)
```

---

### 6. **Spatial Variability Model**
**Soil/basalt chemistry** represented as:
```latex
C_i(x,y) = \mu_i + \sigma_i \cdot \mathcal{N}(0,1)
```
where:
- \(\mu_i\) = Mean concentration of element \(i\)
- \(\sigma_i\) = Standard deviation (spatial variability)
- \((x,y)\) = Spatial coordinates

---

### Summary of Key Parameters
| **Symbol** | **Description** | **Typical Value/Units** |
|------------|-----------------|-------------------------|
| \(k_{\text{diss}}\) | Dissolution rate | \(10^{-8}\) kg/m²/day |
| \(Y\) | Alkalinity yield | 2.0 mol/kg_basalt |
| \(\rho_{\text{soil}}\) | Soil bulk density | 1500 kg/m³ |
| \(d\) | Soil mixing depth | 0.2 m |
| \(K_h\) | Henry's constant | \(3.3 \times 10^{-4}\) mol/L/atm |
| \(K_1\) | Carbonic acid dissociation | \(4.45 \times 10^{-7}\) |
| \(p_{\text{CO}_2}\) | Soil CO₂ partial pressure | 0.015 atm |

These equations form the core mathematical framework of the ERW digital twin, integrating geochemistry, reaction kinetics, and economic constraints. The modular implementation allows customization for different spatial contexts and rock types.
